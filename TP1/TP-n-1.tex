\documentclass[11pt,a4paper]{article}
\usepackage{graphicx}
\usepackage{longtable}
\usepackage{float}
\usepackage{wrapfig}
%\usepackage{rotating}
%\usepackage{a4wide}
%\usepackage[normalem]{ulem}
\usepackage{amsmath}
\usepackage{textcomp}
%\usepackage{marvosym}
%\usepackage{wasysym}
\usepackage{amssymb}
\usepackage{hyperref}
%\usepackage[utf8]{inputenc}
%\usepackage[T1]{fontenc}
\usepackage{polyglossia}
\setmainlanguage[]{french}
\usepackage{listings}

\usepackage[top=30pt,bottom=30pt,left=48pt,right=46pt]{geometry}

\usepackage[dvipsnames]{xcolor}
\usepackage{tikz}
\usetikzlibrary{shadows,shadows.blur, trees,matrix,arrows,decorations}
\usetikzlibrary{decorations.pathmorphing, fadings, shapes, shapes.arrows, positioning, calc, shapes,fit}
\usepackage{fp}
\usepackage{pgfplots}
\tolerance=1000
\author{Xavier JUVIGNY}
\date{\today}
\title{Travaux dirigés n°1}

\definecolor{verylightgray}{rgb}{0.95 0.95 1.0}

\definecolor{darkblue}{rgb}{0. 0. 0.4}


\lstdefinestyle{customcpp}
{
    breaklines=true,
    frame=shadow,
    xleftmargin=\parindent,
    language=C++,
    showstringspaces=false,
    basicstyle=\small\ttfamily,
    keywordstyle=\small\bfseries\color{green!40!black},
    commentstyle=\small\itshape\color{purple!40!black},
    identifierstyle=\small\color{blue},
    stringstyle=\small\color{orange!60!black},
}

\lstset{escapechar=@,style=customcpp}
\newcommand{\includecode}[2][cpp]{
\lstinputlisting[escapechar=, style=custom#1]{#2}
}

\newcommand{\includepartcode}[4][cpp]{
\lstinputlisting[escapechar=, firstline=#3, lastline=#4, style=custom#1]{#2}
}

\begin{document}
\lstset{%
  basicstyle=\scriptsize,
  frame=single,
  keywordstyle=\color{blue},
  language=C++,
  commentstyle=\color{red},
  stringstyle=\color{brown},
  keepspaces=true,
  showspaces=false,
  tabsize=2
}
\lstset{escapechar=@,style=customcpp}

\maketitle
\tableofcontents


\section{Une introduction à MPI} 
\subsection{Une spécification d'interface}

Le standard MPI (\textcolor{blue}{M}essage \textcolor{blue}{P}assing \textcolor{blue}{I}nterface) est une 
\textbf{sp\'ecification} pour les d\'eveloppeurs et
les utilisateurs d'une librairie d'\'echange de messages. En soi, ce n'est pas un bibliothèque, mais plut\^ot
les sp\'ecifications de ce qu'une telle biblioth\`eque devrait \^etre.

MPI propose fondamentalement un \textsl{mod\`ele de programmation parall\`ele par \'echange de messsage} : les donn\'ees sont d\'eplac\'ees
d'un espace d'adressage d'un processus \`a l'espace d'adressage d'un autre processus \`a l'aide d'op\'erations
coop\'eratives sur chacun des processus.

Autrement dit, le but de cet interface est de proposer un standard largement utilis\'e
pour \'ecrire des programmes par \'echanges de messages. L'interface tente d'\^etre :
\begin{itemize}
\item Pratique
\item Portable
\item Optimis\'ee
\item Flexible
\end{itemize}

Les sp\'ecifications de l'interface ont \'et\'e' d\'efinies pour les langages C et Fortran 90. 
La troisi\`eme version des
sp\'ecifications de MPI propose \'egalement un support pour les langages Fortran 2003 et Fortran 2008.

En fait, les diverses bilioth\`eques MPI existantes diff\`erent dans les caract\'eristiques et la version de
sp\'ecification qu'elles supportent. Les utilisateurs de ces biblioth\`eques doivent en avoir conscience.

\subsection{Mod\`ele de programmation}

\`A l'origine, MPI a été con\c{c}u pour des architectures \`a m\'emoires distribu\'ees qui avaient commenc\'e \`a \'emerger \`a
cette \'epoque (fin des ann\'ees 1980 -- début des ann\'ees 1990).

Au fur et \`a mesure du changement des architectures des ordinateurs, les mises en {\oe}uvres de MPI ont adapt\'e leurs
biblioth\`eques aux architectures multi-c{\oe}urs avec m\'emoire partag\'ee et aux diff\'erents protocoles r\'eseaux.

Aujourd'hui, MPI s'ex\'ecute sur virtuellement n'importe quelle plateforme :
\begin{itemize}
\item en m\'emoire distribu\'ee;
\item en m\'emoire partag\'ee;
\item en m\'emoire hybride (distribu\'ee et partag\'ee).
\end{itemize}

N\'eanmoins, le mod\`ele de programmation reste clairement un mod\`ele de \textbf{programmation sur m\'emoire distribu\'ee} sans 
consid\'eration sur l'architecture sous-jacente de la machine.

\textbf{Tout parall\'elisme est explicite} : le programmeur a la responsabilité d'identifier correctement le parall\'elisme
et doit mettre en {\oe}uvre ses algorithmes parall\`eles en utilisant l'interface MPI.

\subsection{Structure g\'en\'erale d'un programme MPI}

\begin{center}
\begin{tikzpicture}
\node[draw,fill=Aquamarine, blur shadow={shadow blur steps=5}] (include) {MPI include file};
\node[below = 2ex of include.south] (decl) {D\'eclarations, prototypes, etc.};
\node[below = 2ex of decl.south] (begin) {D\'ebut du programme};
\node[below = 2ex of begin.east] (serial) {Code s\'equentiel};
\node[below = 4ex of begin.south, draw,fill=Aquamarine, blur shadow={shadow blur steps=5}] (init) {Initialisation de l'environnement MPI};
\node[right = 2em of init.east] (debpar) {\bf Début du code parallèle};
\node[below = 5ex of init.south, draw,fill=Aquamarine, blur shadow={shadow blur steps=5}] (parallel) {Code parall\`ele utilisant les \'echanges de message};
\node[below = 3ex of parallel.south, draw,fill=Aquamarine, blur shadow={shadow blur steps=5}] (finalize) {Terminaison de l'environnement MPI};
\node[right = 3em of finalize.east] (endpar) {\bf Fin du code parall\`ele};
\node[below = 25ex of begin.east] (serial2) {Code s\'equentiel};
\node[below = 4ex of finalize.south] (Fin) {Fin du programme};
\draw[dashed,red] (begin) -- (init);
\draw[dashed,red] (init) -- (parallel);
\draw[dashed,red] (parallel) -- (finalize);
\draw[dashed,red] (finalize) -- (Fin);
\end{tikzpicture}
\end{center}

Le fichier d'entête est nécessaire pour toute utilisation de la bibliothèque MPI :

\begin{lstlisting}[style=customcpp]
# include <mpi.h>
\end{lstlisting}

Les fonctions de l'interface MPI sont de la forme : \texttt{rc = MPI\_Xxxxxxxx(paramètre, paramètre, ....)}.

Par exemple, 
\begin{lstlisting}[style=customcpp]
int rc = MPI_Bsend(&buf, count, type, dest, tag, comm );
\end{lstlisting}

La valeur entière retournée est un code d'erreur. Si aucune erreur n'a été rencontrée, la valeur retournée
vaut \texttt{MPI\_SUCCESS}. Mais, selon le standard MPI, le comportement par défaut d'un appel MPI est
de quitter si il y a une erreur. Cela signifie que vous ne serez pas capable de capturer un code d'erreur
autre que \texttt{MPI\_SUCCESS}. Les types d'erreurs affichés à destination de l'utilisateur dépendent de la mise en
{\oe}uvre de MPI.

\subsection{Communicateurs et groupe}

MPI utilise des objets appelés communicateurs et des groupes qui définissent un ensemble de processus
qui devront communiquer (échanger des messages) entre eux. La majorité des fonctions MPI demandent un
communicateur en argument.

\`A l'initialisation de MPI, un communicateur  est prédéfini : MPI\_COMM\_WORLD. Il inclut tous les
processus créés par MPI. 

Il est possible de définir un nouveau communicateur à partir d'un communicateur existant 
en prenant un sous--ensemble de processus parmi les processus du communicateur existant, mais cela
dépasse le cadre de ce cours.

\section{Fonctions de gestion de l'environnement}

Ces fonctions sont utilisées pour interroger et définir l'environnement d'exécution MPI, et couvrent divers buts comme
l'initialisation et la terminaison de l'environnement MPI, l'interroger de l'identifiant du processus ou la version de MPI utilisée, etc. 

\subsection{Les fonctions de gestion de l'environnement}

Les routines les plus communes sont :

\paragraph{\textcolor{blue}{\texttt{MPI\_Init(\&argc,\&argv)}}}

  Initialise l'environnement d'exécution MPI. Dans un progamme MPI, cette fonction doit être appelée une fois seulement avant toutes autres fonctions MPI. 
  \texttt{MPI\_Init} est utilisé pour passer les arguments de la ligne de commande à tous les processus.

\paragraph{\textcolor{blue}{\texttt{MPI\_Comm\_size(comm, \&size)}}}

  Retourne le nombre total de processus MPI dans le communicateur spécifié.
  Si le communicateur est \texttt{MPI\_COMM\_WORLD}, il représente alors le nombre des tâches MPI disponibles pour votre application.

\paragraph{\textcolor{blue}{\texttt{MPI\_Comm\_rank(comm,\&rank)}}}

  Retourne le rang du processus MPI appelant dans le communicateur spécifié. Initialement, chaque processus se fera
  assigner un entier unique nommé \textsl{rang} entre 0 et le nombre de tâches
  -\ 1 dans le communicateur
  \texttt{MPI\_COMM\_WORLD}. Ce rang est souvent référé comme l'identité de la tâche. Si un processus devient
  associé avec d'autres communicateurs, il aura un rang unique (différent de celui attribué dans
  \texttt{MPI\_COMM\_WORLD}\ !) pour chacun de ces communicateurs.

\paragraph{\textcolor{blue}{\texttt{MPI\_Abort(comm,errocode)}}}

  Termine tous les processus MPI associés avec ce communicateur. Dans la majorité des mises en {\oe}uvre de MPI,
  TOUS les processus sont arrêtés sans considération pour le communicateur.

\paragraph{\textcolor{blue}{\texttt{MPI\_Wtime()}}}

  Retourne le temps réel passé en seconde ( double précision ) sur le processeur appelant.

\paragraph{\textcolor{blue}{\texttt{MPI\_Wtick()}}}

  Retourne la résolution en seconde ( double précision ) de \texttt{MPI\_Wtime}.

\paragraph{\textcolor{blue}{\texttt{MPI\_Finalize()}}}

  Termine l'environnement d'exécution MPI. Cette fonction devra être la dernière fonction appelée dans tout programme MPI -- aucune autre fontion MPI ne devra être appelée après celle-ci.

\subsection{Exemple : Utilisation des fonctions de gestion de l'environnement}

\begin{lstlisting}[style=customcpp]
# include <mpi.h>
# include <cstdlib>

int main( int argc, char *argv[] ) {
  int numtasks, rank;

  // Initialisation de MPI
  MPI_Init( &argc, &argv );

  // Lit le nombre de tâches
  MPI_Comm_size(MPI_COMM_WORLD, &numtasks);

  // Lit mon rang
  MPI_Comm_rank(MPI_COMM_WORLD, &rank);

  // Ici on peut commencer à faire de la programmation
  // parallèle par échange de messages.

  // On termine l'environnement MPI
  MPI_Finalize();

  return MPI_SUCCESS;	
}
\end{lstlisting}

\subsection{Compilation et exécution d'un programme MPI}

Pour compiler un programme utilisant MPI, il est conseillé d'utiliser les outils fourni en standard avec la
bibliothèque. Ces outils vont automatiquement trouver les bons chemins pour les fichiers d'entêtes et les bibliothèques
utilisées par la mise en {\oe}uvre de MPI employée :
\begin{itemize}
\item[\texttt{mpicc}] : Permet de compiler des sources C. Prend les mêmes arguments de compilation que le compilateur local (GNU C sous Linux par exemple).
\item[\texttt{mpiCC}] : Permet de compiler des sources C++. Prend les mêmes arguments de compilation que le compiteur local (GNU C++ sous Linux par exemple).
\item[\texttt{mpif90}] : Permet de compiler des sources Fortran 90. Prend les mêmes arguments de compilation que le compiteur local (GNU Fortran sous Linux par exemple).
\end{itemize}

Pour exécuter en parallèle sur $n$ tâches un programme MPI, on utilise l'exécutable \texttt{mpiexec} fourni en
standard avec MPI. Son utilisation est la suivante :

\texttt{mpiexec [option]} \textbf{$<$program$>$} \texttt{[ $<$args$>$ ]}

où
\begin{itemize}
\item \textbf{$<$program$>$} est le nom de l'exécutable à lancer. Il est identifié comme le premier argument non reconnu par MPI.
\item \texttt{[ $<$args$>$ ]} : Passe les arguments $<$args$>$  à tous les processus créés par MPI. Ils devront toujours être les derniers arguments de mpiexec. 
\item Les options que l'on peut passer à mpiexec sont les suivantes :
\begin{itemize}
\item \texttt{-np $<$n$>$} : Exécute $<$n$>$ copies du programme dans autant de processus. Cette option ne peut être
utilisée que dans un modèle de programmation SPMD;
\item \texttt{-hostfile $<$hostfile$>$} : Utilise le fichier $<$hostfile$>$ pour connaître le nom des machines
( ou des n{\oe}uds de calcul ) sur lesquels on lancera des tâches MPI
\end{itemize}
\end{itemize}

Le format du fichier $<$hostfile$>$ est le suivant :
\begin{verbatim}
aa slots=a1
bb slots=b1
cc slots=c1
\end{verbatim}
où \verb@aa@, \verb@bb@ et \verb@cc@ sont des noms de machines, et \verb@a1@, \verb@a2@ et \verb@a3@ sont des entiers indiquant à MPI le nombre de processus qu'on lui conseille de lancer pour chaque machine.

\subsection{Exercices}

\paragraph{Un hello World parallèle}

\'Ecrire un programme qui, pour chaque processus, affichera à l'écran le message suivant :
\begin{verbatim}
Bonjour, je suis la tâche n° xx sur yy tâches.
\end{verbatim}
où \verb@xx@ est le rang de la tâche et \verb@yy@ le nombre de tâches lancées par MPI.

\section{Fonctions d'échanges de messages point à point}

\subsection{Les types d'opérations point à point}

Les fonctions d'échanges de messages point à point permettent d'échanger des messages
entre deux -- et seulement deux -- tâches. Une des tâches se charge de l'envoi de données
tandis que l'autre tâche se charge de la réception correspondante.

Il y a différents types d'envoi ou de réception, utilisés pour différents buts. Par exemple :
\begin{itemize}
\item les envois synchrones;
\item les envois/réceptions bloquants;
\item les envois/réceptions non bloquants;
\item les envois mis en buffer;
\item les envois/réceptions combinés;
\item les envois ``prêts''.
\end{itemize}

\`A noter que tout type d'envoi peut être reçu avec tout type de réception.

MPI fournit en outre plusieurs routines associées avec les envois et les réceptions telle que celles utilisées
pour tester si un message a bien été reçu, ou si un message est prêt à la réception.

Les données sont transmises via un message, lequel est constitué :
\begin{itemize}
\item \textcolor{blue}{\`A l'envoi} : du numéro de la tâche destinataire, d'un numéro permettant d'identifier
le message (à la réception), des données, de la quantité de donnée et du type de données à envoyer.
\item \textcolor{blue}{\`A la réception} : du numéro de la tâche expéditrice (peut être \texttt{MPI\_ANY\_TAG}), 
du numéro identifiant le message attendu
(mais on peut très bien passer \texttt{MPI\_ANY\_TAG} pour recevoir le premier message provenant de
l'expéditeur), de l'adresse de la zone mémoire où on recevra les données, de la quantité de données
attendues et du type de données attendues.
\end{itemize}

Le type des données est passé à l'aide d'une structure MPI, \texttt{MPI\_Datatype}, qui peut prendre les valeurs données dans le
tableau suivant (avec le type correspondant) :

\begin{center}
\begin{tabular}{|c|c|}\hline
Type MPI & Type C \\ \hline
MPI\_INT & int \\
MPI\_SHORT & short \\
MPI\_LONG & long \\
MPI\_LONG\_LONG\_INT & long long \\
MPI\_UNSIGNED & unsigned \\
MPI\_UNSIGNED\_LONG & unsigned long \\
MPI\_UNSIGNED\_SHORT & unsigned short \\
MPI\_FLOAT & float \\
MPI\_DOUBLE & double \\
MPI\_LONG\_DOUBLE & long double \\
MPI\_CHAR & char \\
MPI\_UNSIGNED\_CHAR & unsigned char \\
MPI\_C\_COMPLEX & float\_Complex \\
MPI\_C\_DOUBLE\_COMPLEX & double\_Complex \\
MPI\_BYTE & void* \\
MPI\_PACKED & données hétérogènes\\ \hline
\end{tabular}
\end{center}

\subsection{Buffering}

Dans un monde parfait, les envois devraient être parfaitement synchronisés avec les réceptions correspondantes, mais 
en réalité c'est rarement le cas. D'une certaine manière, la bibliothèque MPI utilisée doit pouvoir gérer les données stockées lorsque les deux tâches sont non synchrones. 

Par exemple :
\begin{itemize}
\item Une opération d'envoi a lieu cinq secondes avant que la réception soit prête -- où se trouve le message avant que la
réception soit faite ?
\item Plusieurs envois arrivent simultanément sur la même tâche qui ne peut accepter qu'une réception à la fois -- que deviennent
les messages en attente de réception ?
\end{itemize}

La mise en {\oe}uvre de la bibliothèque MPI utilisée décide de la gestion de ces données dans tous les cas qui se présentent
(cette gestion ne fait pas partie du standard !). Typiquement, une zone  \textbf{buffer système} est réservée pour contenir les données
en transit. 

\begin{center}
\begin{tikzpicture}
\draw[draw=none, fill=cyan!75!black, drop shadow=blue!25!black, very thick] (-6cm,-6cm) rectangle (-2cm,0cm);
\draw[very thick,color=cyan] (-6cm,-6cm) -- (-6cm,0cm) -- (-2cm,0cm);
\draw[very thick,color=cyan!50!black] (-6cm,-6cm) -- (-2cm,-6cm) -- (-2cm,0cm);
\node[minimum width=8, centered] at (-4cm,-1em) {\Large N{\oe}ud 1};
\draw[draw=none,fill=blue!75!black, drop shadow=blue!20!black, very thick] (-5.5cm,-5.5cm) rectangle (-2.5cm,-1cm);
\draw[color=blue, very thick] (-5.5cm,-5.5cm) -- (-5.5cm,-1cm) -- (-2.5cm,-1cm);
\draw[color=blue!50!black, very thick] (-5.5cm,-5.5cm) -- (-2.5cm,-5.5cm) -- (-2.5cm,-1cm);
\node[centered] at (-4cm,-1cm-1em) {\bf Tâche A};
\draw[fill=white,draw] (-5.25cm,-3cm) rectangle (-2.75cm, -1.6cm);
\node[centered] at (-4cm,-1.6cm-1em) (send) {\small Send};
\node[rectangle,fill=violet!50!white, drop shadow=black] at (-4cm,-2.5cm) (ds) {\scriptsize data};
\draw[fill=white,draw] (-5.25cm,-3.5cm) rectangle (-2.75cm, -5.25cm);
\node[centered] at (-4cm,-3.5cm-1em) {\small System Buffer};


\draw[draw=none, fill=cyan!75!black, drop shadow=blue!25!black, very thick] (+2cm,-6cm) rectangle (+6cm,0cm);
\draw[thick,color=cyan] (+2cm,-6cm) -- (+2cm,0cm) -- (+6cm,0cm);
\draw[thick,color=cyan!50!black] (+2cm,-6cm) -- (+6cm,-6cm) -- (+6cm,0cm);
\node[minimum width=8, centered] at (+4cm,-1em) {\Large N{\oe}ud 2};
\draw[draw=none,fill=blue!75!black, drop shadow=blue!20!black, very thick] (+2.5cm,-5.5cm) rectangle (+5.5cm,-1cm);
\draw[color=blue, very thick] (+2.5cm,-5.5cm) -- (+2.5cm,-1cm) -- (+5.5cm,-1cm);
\draw[color=blue!50!black, very thick] (+2.5cm,-5.5cm) -- (+5.5cm,-5.5cm) -- (+5.5cm,-1cm);
\node[centered] at (+4cm,-1cm-1em) {\bf Tâche B};
\draw[fill=white,draw] (5.25cm,-3cm) rectangle (2.75cm, -1.6cm);
\node[centered] at (4cm,-1.6cm-1em) (send) {\small Recv};
\node[rectangle,fill=violet!50!white, drop shadow=black] at (4.25cm,-2.5cm) (dr2) {\scriptsize data};
\draw[fill=white,draw] (5.25cm,-3.5cm) rectangle (2.75cm, -5.25cm);
\node[centered] at (4cm,-3.5cm-1em) {\small System Buffer};
\node[rectangle,fill=violet!50!white, drop shadow=black] at (3.5cm,-4.5cm) (dr1) {\scriptsize data};
\node at (4.25cm, -4.5cm) (ph) {};
\node at (0cm,-2cm) {$\leftarrow$ Réseau $\rightarrow$};
\draw[red,->] (ds) -- (dr1);
\draw[red,->] (dr1.east) -| (dr2.south);
\end{tikzpicture}
\end{center}

Cette zone privée est :
\begin{itemize}
\item opaque au programmeur et entièrement gérée par la bibliothèque MPI employée;
\item une ressource finie qui peut être facilement épuisée;
\item souvent mystérieuse et mal documentée;
\item capable d'exister du côté de l'envoi, de la réception ou des deux à la fois;
\item quelque chose qui devrait améliorer les performances du programme car permettant l'envoi et
la réception en asynchrone. 
\end{itemize}

L'espace d'adresse des variables du programme est appelée le \textbf{buffer d'application}. MPI permet à l'utilisateur de gérer son
buffer d'envoi dans cet espace.

\subsection{Bloquant contre non bloquant}

La plupart des routines point à point de MPI peuvent être utilisées en communication bloquante ou non bloquante. 

\begin{itemize}
\item[Bloquant]
  \begin{itemize}
  \item Un envoi bloquant rendra la main seulement après s'être assuré que modifier le buffer d'application est ``sûr''.
``Sûr'' signifie ici que ces modifications n'altérera pas les données qui devront être reçues par la tâche destinataire -- elles peuvent
par contre être stockées dans un buffer système.
\item Un envoi bloquant peut être synchrone, ce qui signifie qu'il y a un échange d'information avec la tâche destinataire pour
confirmer que l'envoi est sûr;
\item Un envoi bloquant peut être asynchrone si un buffer système est utilisé pour contenir les données à envoyer au destinataire;
\item Une réception bloquante rend la main seulement après que les données soient arrivées et prêtes à être utilisées par
l'utilisateur.
  \end{itemize}
\item[Non bloquant]
\begin{itemize}
\item Les envois et réceptions non bloquants se comportent de manière similaire -- ils rendent la main immédiatement. Ils n'attendent pas
qu'une étape de communication soit finie, telle que la copie du buffer d'application vers le buffer système ou l'arrivée du message attendu;
\item Les opérations non bloquantes interrogent simplement la bibliothèque MPI pour exécuter les opérations quand elles sont valides. L'utilisateur
ne peut pas deviner le moment où cela va se passer;
\item Il est dangereux de modifier le buffer d'application tant que vous n'êtes pas certain que l'opération non bloquante demandée se soit
exécutée. Il existe des routines de type ``attente'' pour s'en assurer.
\item Les communications non bloquantes sont utilisées surtout pour recouvrir les communications avec des calculs et pouvoir ainsi gagner
en rapidité d'exécution.
\end{itemize}
\end{itemize}

\begin{minipage}{8cm}
\begin{center}Envoi bloquant\end{center}

\begin{lstlisting}[style=customcpp]
myvar = 0;
for ( i = 1; i < numtasks, ++i ) {
  task = i;
  MPI_Send(&myvar,...,...,task,...);
  myvar = myvar + 2;
  // Do some works
  ...
  ...
}
\end{lstlisting}

\begin{center}Sûr. Pourquoi ?\end{center}
\end{minipage}
\begin{minipage}{8cm}
\begin{center}Envoi non bloquant\end{center}

\begin{lstlisting}[style=customcpp]
myvar = 0;
for ( i = 1; i < numtasks, ++i ) {
  task = i;
  MPI_ISend(&myvar,...,...,task,...);
  myvar = myvar + 2;
  // Do some works
  ...
  MPI_Wait (...);
}
\end{lstlisting}

\begin{center}Dangereux. Pourquoi ?\end{center}
\end{minipage}

\subsection{Routines d'échange de message MPI}

Les communications points à points MPI prennent en général une liste d'argument similaire.\\
{\small
\begin{tabular}{|c|c|}\hline
\textbf{Envoi bloquant} & \texttt{MPI\_Send(buffer, count, type, dest, idtag, comm)} \\ \hline
\textbf{Envoi non bloquant} & \texttt{MPI\_ISend(buffer, count, type, dest, idtag, comm, request)} \\ \hline
\textbf{Réception bloquante} & \texttt{MPI\_Recv(buffer, count, type, source, idtag, comm, status)} \\ \hline
\textbf{Réception non bloquante} & \texttt{MPI\_IRecv(buffer, count, type, source, idtag, comm, request)} \\ \hline
\end{tabular}
}\\
où
\begin{itemize}
\item[{\bf buffer}] Adresse référençant les variables à envoyer ou recevoir. Dans la majorité des cas, c'est simplement le nom
de la variable qui doit être envoyée ou dans laquelle on reçoit. Si c'est un simple scalaire, on passe la variable par adresse,
ce qui se traduit en C par un symbole \& avant la variable.
\item[{\bf count}] Indique dans le cas d'un tableau le nombre d'éléments du tableau à envoyer.
\item[{\bf type}]  Pour des raisons de portabilité, MPI prédéfinit des types de données élémentaires. Voir la table donnée préalablement
pour les types fournis par MPI.
\item[{\bf dest}] Un argument pour les routines d'envoi spécifiant le numéro de la tâche destinée à recevoir le message;
\item[{\bf source}] Spécifie le numéro de la tâche dont on attend le message. Ce numéro peut être un ``joker'' et on peut attendre un message
provenant de n'importe quelle tâche grâce à la variable \texttt{MPI\_ANY\_SOURCE}.
\item[{\bf idtag}] Un entier arbitraire donné par le programmeur pour identifier son message. Ce nombre peut être compris entre 0
et 32767. En réception, la routine attend un message portant cet identifiant pour le recevoir. On peut là encore utiliser un ``joker''
à l'aide de la variable \texttt{MPI\_ANY\_TAG};
\item[{\bf comm}] Le communicateur pour lequel les numéros des tâches utilisées sont valides. Sauf si le programmeur a créé des communicateurs,
on utilise par défaut \texttt{MPI\_COMM\_WORLD}.
\item[{\bf status}] Pour une opération de réception, indique la source et l'identifiant du message. En C, c'est un pointeur sur une structure
de type \texttt{MPI\_Status}. 
\item[{\bf request}] Pour les opérations non bloquantes, c'est une structure opaque servant à analyser l'état de l'envoi ou de la réception.
Cette variable sera utilisée ensuite pour synchroniser la réception ou l'envoi (à l'aide d'une routine de type \texttt{MPI\_Wait}).
\end{itemize}

\paragraph{Fonctions d'échange de message bloquantes}

\begin{itemize}
\item \textcolor{blue}{\texttt{MPI\_Send}} :

C'est l'opération basique d'envoi bloquant. La fonction ne rend la main que lorsque le buffer d'application dans la tâche est de nouveau libre
pour être reutilisé. 

\begin{lstlisting}[style=customcpp]
MPI_Send (&buf,count,datatype,dest,tag,comm) 
\end{lstlisting}

\underline{Exemples} :
\begin{lstlisting}[style=customcpp]
int myval = 3;
double mybuf[100] = { ... };
MPI_Send(&myval, 1, MPI_INT, rank+1, 0, MPI_COMM_WORLD);
MPI_Send(mybuf, 100, MPI_DOUBLE, rank+1, 0, MPI_COMM_WORLD );
\end{lstlisting}

\item \textcolor{blue}{\texttt{MPI\_Recv}}

Reçoit un message. Ne rend pas la main jusqu'à ce que le buffer d'application  soit utilisable par l'utilisateur.

\begin{lstlisting}[style=customcpp]
MPI_Recv (&buf,count,datatype,source,tag,comm, &status) 
\end{lstlisting}


Exemples :
\begin{lstlisting}[style=customcpp]
int val;
double buf[100];
MPI_Status status;
MPI_Recv(&val, 1, MPI_INT, rank-1, 0, MPI_COMM_WORLD,&status);
MPI_Recv(buf, 100, MPI_DOUBLE, rank-1, 0, MPI_COMM_WORLD, &status );
\end{lstlisting}

\item \textcolor{blue}{\texttt{MPI\_Ssend}}

Envoi bloquant synchrone. Envoie un message et attend que le buffer d'application dans la tâche émettrice
puisse de nouveau être reutilisé et que la tâche destinataire ait commencé à recevoir le message.

\begin{lstlisting}[style=customcpp]
MPI_Ssend( &buf, count, datatype, dest, tag, comm)
\end{lstlisting}

Exemples :
\begin{lstlisting}[style=customcpp]
int val;
double buffer[255];
MPI_Ssend( &val, 1, MPI_INT, rank+1, 101, MPI_COMM_WORLD );
MPI_Ssend( buffer, 255, MPI_DOUBLE, rank+1, 30, MPI_COMM_WORLD );
\end{lstlisting}

\item \textcolor{blue}{\texttt{MPI\_Sendrecv}}

Envoie un message puis attend une réception avant de bloquer. Bloque jusqu'à ce que le buffer d'application
d'envoi soit de nouveau disponible pour utilisation et que le buffer d'application de réception contienne
le message reçu.

\begin{lstlisting}[style=customcpp]
MPI_Sendrecv( &sendBuf, sendCount, sendType, dest, sendTag, 
              &recvBuf, recvCount, recvType, source, recvTag,
              comm, &status );
\end{lstlisting}

Exemples :
\begin{lstlisting}[style=customcpp]
int sendVal;
double sendBuf[255];
int recvVal;
double recvBuf[255];
MPI_Status status;
MPI_Sendrecv( &sendVal, 1, MPI_INT, rank+1, 101,
              &recvVal, 1, MPI_INT, rank-1, 101,
              MPI_COMM_WORLD, &status );
MPI_Sendrecv( sendBuf, 1, MPI_DOUBLE, rank+1, 101,
              recvBuf, 1, MPI_DOUBLE, rank-1, 101,
              MPI_COMM_WORLD, &status ); 
\end{lstlisting}

\item \textcolor{blue}{\texttt{MPI\_Probe}}

Effectue un test bloquant sur un message. Les ``jokers'' \texttt{MPI\_ANY\_SOURCE} et \texttt{MPI\_ANY\_TAG} peuvent être utilisés
pour tester un message provenant de n'importe quelle tâche ou identifié par n'importe quel nombre. \`A la sortie de la fonction,
le numéro de l'expéditeur et le numéro identifiant le message seront retournés dans le paramètre \texttt{status} (à l'aide
de \texttt{status.MPI\_SOURCE} et \texttt{status.MPI\_TAG}).

\begin{lstlisting}[style=customcpp]
MPI_Probe( source, tag, comm, &status );
\end{lstlisting}

\item \textcolor{blue}{\texttt{MPI\_Get\_count}}

Retourne le numéro de l'expéditeur, l'identifiant du message et le nombre d'éléments d'un certain type reçu. Peut être utilisée avec des opérations
de réception bloquantes ou non. Le numéro de l'expéditeur et l'identifiant seront retournés dans la structure de \texttt{status} comme 
\texttt{status.MPI\_SOURCE} et \texttt{status.MPI\_TAG}. 

\begin{lstlisting}[style=customcpp]
MPI_Get_count( &status, datatype, &count )
\end{lstlisting}

\end{itemize}

\underline{Exemple d'utilisation des fonctions bloquantes d'échange de messages}

\begin{lstlisting}[style=customcpp]
#include "mpi.h"
#include <stdio.h>

int main(int argc, char *argv[])  {
  int numtasks, rank, dest, source, rc, count, tag=1;  
  char inmsg, outmsg='x';
  MPI_Status Stat;   // required variable for receive routines

  MPI_Init(&argc,&argv);
  MPI_Comm_size(MPI_COMM_WORLD, &numtasks);
  MPI_Comm_rank(MPI_COMM_WORLD, &rank);

  // task 0 sends to task 1 and waits to receive a return message
  if (rank == 0) {
    dest = 1;
    source = 1;
    MPI_Send(&outmsg, 1, MPI_CHAR, dest, tag, MPI_COMM_WORLD);
    MPI_Recv(&inmsg, 1, MPI_CHAR, source, tag, MPI_COMM_WORLD, &Stat);
  } 
  // task 1 waits for task 0 message then returns a message
  else if (rank == 1) {
    dest = 0;
    source = 0;
    MPI_Recv(&inmsg, 1, MPI_CHAR, source, tag, MPI_COMM_WORLD, &Stat);
    MPI_Send(&outmsg, 1, MPI_CHAR, dest, tag, MPI_COMM_WORLD);
  }

  // query recieve Stat variable and print message details
  MPI_Get_count(&Stat, MPI_CHAR, &count);
  printf("Task %d: Received %d char(s) from task %d with tag %d \n",
         rank, count, Stat.MPI_SOURCE, Stat.MPI_TAG);

  MPI_Finalize();
}
\end{lstlisting}

\paragraph{Fonctions non bloquantes d'échange de message}

Les fonctions non bloquantes les plus utilisées pour échanger des messages
sont données ci--dessous. On y donne également les fonctions bloquantes
qui leur sont associées.

\begin{itemize}
\item \textcolor{blue}{\texttt{MPI\_Isend}}

Utilise une aire en mémoire pour servir comme buffer d'envoi. La tâche continue
immédiatement après l'appel sans attendre que le message soit copié du
buffer d'application. Une requête est renvoyée permettant au programmeur de
connaître l'état de l'envoi. Le programmeur doit s'assurer de ne pas modifier
le buffer d'application jusqu'à ce qu'il ait testé si le buffer d'application
est de nouveau prêt à l'emploi grâce aux fonctions \texttt{MPI\_Wait} ou
\texttt{MPI\_Test}.

\begin{lstlisting}[style=customcpp]
MPI_Isend( &buf, count, datatype, dest, tag, comm, &request )
\end{lstlisting}

\item \textcolor{blue}{\texttt{MPI\_Irecv}}

Utilise une aire en mémoire pour servir de buffer de réception. La tâche continue
immédiatement après l'appel à la fonction sans attendre que le message soit copié
dans le buffer d'application. Une requête est renvoyée permettant au programmeur
de connaître l'état de la réception. Le programmeur doit utiliser une des
fonctions \texttt{MPI\_Wait} ou \texttt{MPI\_Test} pour s'assurer que le message
reçu est valable dans le buffer d'application.

\begin{lstlisting}[style=customcpp]
MPI_Irecv( &buf, count, datatype, source, tag, comm, &request )
\end{lstlisting}

\item \textcolor{blue}{\texttt{MPI\_Issend}}

Envoi synchrone non bloquant. Cette fonction est similaire à \texttt{MPI\_Isend}
sauf que \texttt{MPI\_Wait} ou \texttt{MPI\_Test} indiquent si la tâche destinataire
a reçu ou non le message.

\begin{lstlisting}[style=customcpp]
MPI_Issend( &buf, count, datatype, dest, tag, comm, &request )
\end{lstlisting}

\item \textcolor{blue}{\texttt{MPI\_Test}, \texttt{MPI\_Testany}, \texttt{MPI\_Testall}, \texttt{MPI\_Testsome}}

Ces fonctions testent le status d'un envoi ou d'une réception non bloquante spécifique à l'aide de la requête associée. 
Le paramêtre ``\texttt{flag}'' est retourné
comme logiquement vrai (=1) si l'opération est finie et logiquement fausse (=0) sinon. Pour plusieurs opérations non
bloquantes, le programmeur peut demander à tester si une des opérations, toutes les opérations ou certaines opérations
sont finies.

\begin{lstlisting}[style=customcpp]
MPI_Test(&request, &flag, &status );
MPI_Testany( count, &array_of_requests, &index, &flag, &status );
MPI_Testall( count, &array_of_requests, &flag, &array_of_statuses );
MPI_Testsome( incount, &array_of_requests, &outcount, &array_of_offsets, 
              &array_of_statuses );
\end{lstlisting}

\item \textcolor{blue}{\texttt{MPI\_Wait}, \texttt{MPI\_Waitany}, \texttt{MPI\_Waitall}, \texttt{MPI\_Waitsome}}

Ces fonctions bloquent jusqu'à ce qu'un envoi ou une réception spécifié soit fini. Pour plusieurs envois/réceptions
non bloquants, le programmeur peut spécifier si une des opérations, toutes ou certaines opérations sont finies.

\begin{lstlisting}[style=customcpp]
MPI_Wait(&request, &status );
MPI_Waitany( count, &array_of_requests, &index, &status );
MPI_Waitall( count, &array_of_requests, &array_of_statuses );
MPI_Waitsome( incount, &array_of_requests, &outcount, &array_of_offsets, 
              &array_of_statuses );
\end{lstlisting}

\item \textcolor{blue}{\texttt{MPI\_Iprobe}}

Effectue un test non bloquant pour un message. Les ``jokers'' \texttt{MPI\_ANY\_SOURCE} et \texttt{MPI\_ANY\_TAG}
peuvent être utilisés pour testé un message provenant de n'importe quelle autre tâche ou ayant n'importe
quel identifiant. Le paramêtre entier ``\texttt{flag}'' est retourné avec une valeur logiquement vraie ( = 1 )
si le message est arrivé et logiquement fausse sinon ( = 0 ). Le numéro de l'expéditeur et l'identifiant du
message sont retournés via \texttt{status} comme \texttt{status.MPI\_SOURCE} et \texttt{status.MPI\_TAG}.

\begin{lstlisting}[style=customcpp]
MPI_Iprobe( source, tag, comm, &flag, &status )
\end{lstlisting}

\end{itemize}

\underline{Exemple de programme utilisant des messages non bloquants}

\'Echange de données avec son plus proche voisin dans une topologie en anneau :

\begin{center}
\begin{tikzpicture}
  \tikzset{VertexStyle/.style = { shape        = circle,
                                  ball color   = orange,
                                  text         = black,
                                  inner sep    = 2pt,
                                  outer sep    = 0pt,
                                  minimum size = 24pt} }
  \tikzset{EdgeStyle/.style   = {thick,
                                 double        = orange,
                                 double distance = 1pt,rounded corners=5pt } }

  \node[VertexStyle](N0){0};
  \node[VertexStyle,right =of N0](N1){1};
  \node[VertexStyle,right =of N1](N2){2};
  \node[VertexStyle,right =of N2](N3){3};
  \node[VertexStyle,right =of N3](Np){\ldots};
  \node[VertexStyle,right =of Np](Nn){N-1};
  \draw[EdgeStyle,->] (N0.south east) -- (N1.south west);
  \draw[EdgeStyle,->] (N1.north west) -- (N0.north east);
  \draw[EdgeStyle,->] (N1.south east) -- (N2.south west);
  \draw[EdgeStyle,->] (N2.north west) -- (N1.north east);
  \draw[EdgeStyle,->] (N2.south east) -- (N3.south west);
  \draw[EdgeStyle,->] (N3.north west) -- (N2.north east);
  \draw[EdgeStyle,->] (N3.south east) -- (Np.south west);
  \draw[EdgeStyle,->] (Np.north west) -- (N3.north east);
  \draw[EdgeStyle,->] (Np.south east) -- (Nn.south west);
  \draw[EdgeStyle,->] (Nn.north west) -- (Np.north east);
  \coordinate[shift={(0mm,5mm)}] (s0) at (Nn.north);
  \coordinate[shift={(0mm,-5mm)}] (r0) at (Nn.south);
  \draw[EdgeStyle,->] (Nn.north) -| (s0) -| (N0.north);
  \draw[EdgeStyle,->] (N0.south) |- (r0) -| (Nn.south);
\end{tikzpicture}
\end{center}

\begin{lstlisting}[style=customcpp]
#include "mpi.h"
#include <stdio.h>

int main(int argc, char *argv[])  {
   int numtasks, rank, next, prev, buf[2], tag1=1, tag2=2;
   MPI_Request reqs[4];   // required variable for non-blocking calls
   MPI_Status stats[4];   // required variable for Waitall routine

   MPI_Init(&argc,&argv);
   MPI_Comm_size(MPI_COMM_WORLD, &numtasks);
   MPI_Comm_rank(MPI_COMM_WORLD, &rank);
   
   // determine left and right neighbors
   prev = rank-1;
   next = rank+1;
   if (rank == 0)  prev = numtasks - 1;
   if (rank == (numtasks - 1))  next = 0;

   // post non-blocking receives and sends for neighbors
   MPI_Irecv(&buf[0], 1, MPI_INT, prev, tag1, MPI_COMM_WORLD, &reqs[0]);
   MPI_Irecv(&buf[1], 1, MPI_INT, next, tag2, MPI_COMM_WORLD, &reqs[1]);

   MPI_Isend(&rank, 1, MPI_INT, prev, tag2, MPI_COMM_WORLD, &reqs[2]);
   MPI_Isend(&rank, 1, MPI_INT, next, tag1, MPI_COMM_WORLD, &reqs[3]);
  
   // do some work while sends/receives progress in background

   // wait for all non-blocking operations to complete
   MPI_Waitall(4, reqs, stats);
  
   // continue - do more work

   MPI_Finalize();
}
\end{lstlisting}

\subsection{Exercices}

\paragraph{Circulation d'un jeton dans un anneau}

On veut faire circuler un jeton dans un réseau topologiquement équivalent
à un anneau (un exemple de jeton est un fichier de licence).

L'algorithme de circulation du jeton pour \texttt{nbp} tâches est le suivant :
\begin{enumerate}
\item La première tâche (tâche 0) génère un entier (dont la valeur est choisi arbitrairement par le
  programmeur, et considéré ici comme le jeton), et l'envoie au processus suivant (le processus 1);
  Il reçoit ensuite le jeton du processus \texttt{nbp}-1 et affiche sa valeur;
\item Le processus 1 reçoit le jeton, incrémente sa valeur de 1 et l'envoie au prochain processus;
\item \ldots
\item Le processus $p$ reçoit le jeton, l'incrémente de 1 et l'envoie au processus suivant
(processus $p+1$);
\item \ldots
\item Le dernier processus (processus \texttt{nbp}-1) reçoit le jeton, l'incrémente de 1 et
  l'envoie au processus 0;
\end{enumerate}

En vous servant de l'exercice sur l'environnement MPI, mettez en {\oe}uvre l'algorithme décrit ci--dessus, puis
compilez et testez sur divers nombres de tâches.

\fbox{
\begin{minipage}{\textwidth}
\underline{Astuce} : \\
\begin{enumerate}
 \item Lorsque vous programmez en parallèle, essayez de penser au niveau local d'un processus et non au niveau global (sur tous les processus).
 \item Pour déboguer votre programme, n'hésitez pas à écrire des messages avant et après envoi/réception de messages, suivit d'un
  appel à \texttt{fflush}. En effet, il arrive souvent lors de la mise au point d'un programme en parallèle de se retrouver
  dans une situation de \textsl{dead--lock}, c'est à dire dans une situation où un ou plusieurs processus attendent indéfiniment un message
  en réception alors que ce message ne sera jamais envoyé.
\end{enumerate} 
\end{minipage}}

\paragraph{Calcul de pi par lancer de fléchettes}

On veut calculer la valeur de pi à l'aide de l'algorithme stochastique suivant :
\begin{itemize}
\item On considère le carré unité
$\left[-1;1\right]\times \left[-1;1\right]$ dans lequel on inscrit le cercle unité de centre $(0,0)$ et de rayon $1$.
\item On génère des points aléatoirement dans le carré unité.
\item On compte le nombre de points générés dans le carré qui sont aussi dans le cercle.
\item Soit $r$ ce nombre de points dans le cercle divisé par le nombre de points
  total dans le carré, on calcule alors pi comme $\pi = 4.r$.
\end{itemize}

Remarquez que l'erreur faite sur pi décroit quand le nombre de points générés augmente.

\begin{center}
\begin{tikzpicture}[scale=2]
\draw[color=black,fill=cyan] (0,0) circle (1);
\draw[color=black] (-1,-1) -- (-1,1) -- (1,1) -- (1,-1) -- cycle;
\draw[red] plot[only marks, mark=*, mark size=0.5, domain=-1:1, samples=400] (\x,{rnd*2-1});
\end{tikzpicture}
\end{center}

La fonction C++ permettant de générer les points dans le carré et calculer le ratio entre nombre
de points dans le cercle et nombre de points générés est la suivante :

\begin{lstlisting}[style=customcpp]
# include <chrono>
# include <random>
// Attention, ne marche qu'en C++ 11 ou supérieur :
double approximate_pi( unsigned long nbSamples ) {
  typedef std::chrono::high_resolution_clock myclock;
  myclock::time_point beginning = myclock::now();

  myclock::duration d = myclock::now() - beginning;
  unsigned seed = d.count();
  std::default_random_engine generator(seed);
  std::uniform_real_distribution<double> distribution(-1.0,1.0);
  unsigned long nbDarts = 0;
  // Throw nbSamples darts in the unit square [-1:1] x [-1:1]
  for ( unsigned sample = 0; sample < nbSamples; ++sample ) {
    double x = distribution(generator);
    double y = distribution(generator);
    // Test if the dart is in the unit disk
    if ( x*x+y*y<=1 ) nbDarts ++;
  }
  // Number of nbDarts throwed in the unit disk
  double ratio = double(nbDarts)/double(nbSamples);
  return ratio;
}
\end{lstlisting}

\begin{itemize}
\item Couper l'itération de boucle en plusieurs morceaux pouvant être exécutés par différentes tâches
simultanément;
\item Chaque tâche exécute sa portion de boucle;
\item Chaque tâche peut exécuter son travail sans avoir besoin d'information des autres tâches ( indépendance
des données );
\item La tâche maître ( que le programmeur aura choisi parmi ses tâches ) reçoit le résultat des autres
tâches à l'aide d'échanges de message point à point.
\item Dans un deuxième temps, essayer à l'aide de réceptions non bloquantes de recouvrir les réceptions par du
calcul dans le processus maître.
\end{itemize}

\begin{center}
\begin{tikzpicture}[scale=2]
\draw[color=black,fill=cyan] (0,0) circle (1);
\draw[color=black] (-1,-1) -- (-1,1) -- (1,1) -- (1,-1) -- cycle;
\draw[color=red,fill=red] plot[only marks, mark=*, mark size=0.5, domain=-1:1, samples=100] (\x,{rnd*2-1});
\draw[color=green,fill=green] plot[only marks, mark=*, mark size=0.5, domain=-1:1, samples=100] (\x,{rnd*2-1});
\draw[color=yellow, fill=yellow] plot[only marks, mark=*, mark size=0.5, domain=-1:1, samples=100] (\x,{rnd*2-1});
\draw[color=violet, fill=violet] plot[only marks, mark=*, mark size=0.5, domain=-1:1, samples=100] (\x,{rnd*2-1});
\draw (-0.75,-1.5) node[right,rectangle,fill=red] (t1) {task 1};
\draw node[right, right=of t1, rectangle,fill=green] (t2) {task 2};
\draw node[right, below=of t1, rectangle, fill=yellow] (t3) {task 3};
\draw node[right, right=of t3, rectangle, fill=violet] (t4) {task 4};
\end{tikzpicture}
\end{center}

\fbox{
\begin{minipage}{\textwidth}
\underline{Le concept} : diviser le travail parmi les tâches disponibles en communicant des données à l'aide
d'appel à des fonctions d'envoi/réception point à point
\end{minipage} }

\paragraph{Diffusion d'un entier dans un réseau hypercube\textcolor{red}{$^{\star}$}}

On veut écrire un programme qui diffuse un entier dans un réseau de n{\oe}uds de calculs dont la topologie
est équivalente à celle d'un hypercube de dimension $d$ ( et qui contient donc $2^{d}$ n{\oe}uds de calcul ).

\begin{enumerate}
\item \'Ecrire un programme qui diffuse un entier dans un hyper cube de dimension 1 :
\begin{itemize}
\item La tâche 0 initialise un jeton à une valeur entière choisie par le programmeur et envoie cette valeur 
à la tâche 1; 
\item La tâche 1 reçoit la valeur du jeton de la tâche 0;
\item Les deux tâches affichent la valeur du jeton;
\end{itemize}
\item Diffuser le jeton généré par la tâche 0 dans un hypercube de dimension 2 de manière que cette diffusion se fasse en un minimum
d'étapes ( et donc un maximum de communications simultanées entre tâches ).
\item Faire de même pour un hypercube de dimension 3
\item \'Ecrire le cas général quand le cube est de dimension $d$. Le nombre d'étapes pour diffuser le jeton devra
être égal à la dimension de l'hypercube.
\end{enumerate}

\begin{center}
\begin{tikzpicture}
% Cube dimension 0
\draw (-4,0) node[fill=green,inner sep=1pt]{0};
\draw (-4,-0.5) node{dim. 0};
% Cube dimension 1
\draw (-3,0) node[fill=green,inner sep=1pt](cd1_0){0} -- (-2,0) node[fill=cyan,inner sep=1pt](cd1_1){1};
\draw (-2.5,-0.5) node{dim. 1};
% Cube dimension 2
\draw (-1,0) node[fill=green,inner sep=1pt](cd2_0){0} -- (0,0) node[fill=green,inner sep=1pt](cd2_1){1};
\draw (-1,1) node[fill=cyan,inner sep=1pt](cd2_2){2} -- (0,1) node[fill=cyan,inner sep=1pt](cd2_3){3};
\draw (cd2_0) -- (cd2_2) (cd2_1) -- (cd2_3);
\draw (-0.5,-0.5) node{dim. 2};
% Cube dimension 3
\draw (1,0) node[fill=green,inner sep=1pt](cd3_0){0} -- (2,0) node[fill=green,inner sep=1pt](cd3_1){1};
\draw (1,1) node[fill=green,inner sep=1pt](cd3_2){2} -- (2,1) node[fill=green,inner sep=1pt](cd3_3){3};
\draw (1.5,0.5) node[fill=cyan,inner sep=1pt](cd3_4){4} -- (2.5,0.5) node[fill=cyan,inner sep=1pt](cd3_5){5};
\draw (1.5,1.5) node[fill=cyan,inner sep=1pt](cd3_6){6} -- (2.5,1.5) node[fill=cyan,inner sep=1pt](cd3_7){7};
\draw (cd3_0) -- (cd3_2) (cd3_1) -- (cd3_3);
\draw (cd3_4) -- (cd3_6) (cd3_5) -- (cd3_7);
\draw (cd3_0) -- (cd3_4) (cd3_1) -- (cd3_5) (cd3_2)--(cd3_6) (cd3_3)--(cd3_7);
\draw (1.5,-0.5) node{dim. 3};
% Cube de dimension 4
\draw (6,0) node[fill=green,inner sep=1pt](cd4_0){0} -- (7,0) node[fill=green,inner sep=1pt](cd4_1){1};
\draw (6,1) node[fill=green,inner sep=1pt](cd4_2){2} -- (7,1) node[fill=green,inner sep=1pt](cd4_3){3};
\draw (6.5,0.5) node[fill=green,inner sep=1pt](cd4_4){4} -- (7.5,0.5) node[fill=green,inner sep=1pt](cd4_5){5};
\draw (6.5,1.5) node[fill=green,inner sep=1pt](cd4_6){6} -- (7.5,1.5) node[fill=green,inner sep=1pt](cd4_7){7};
\draw (cd4_0) -- (cd4_2) (cd4_1) -- (cd4_3);
\draw (cd4_4) -- (cd4_6) (cd4_5) -- (cd4_7);
\draw (cd4_0) -- (cd4_4) (cd4_1) -- (cd4_5) (cd4_2)--(cd4_6) (cd4_3)--(cd4_7);
\draw (4,0.5) node[fill=cyan,inner sep=1pt](cd4_8){8} -- (5,0.5) node[fill=cyan,inner sep=1pt](cd4_9){9};
\draw (4,1.5) node[fill=cyan,inner sep=1pt](cd4_10){10} -- (5,1.5) node[fill=cyan,inner sep=1pt](cd4_11){11};
\draw (4.5,1) node[fill=cyan,inner sep=1pt](cd4_12){12} -- (5.5,1) node[fill=cyan,inner sep=1pt](cd4_13){13};
\draw (4.5,2) node[fill=cyan,inner sep=1pt](cd4_14){14} -- (5.5,2) node[fill=cyan,inner sep=1pt](cd4_15){15};
\draw (cd4_8) -- (cd4_10) (cd4_9) -- (cd4_11);
\draw (cd4_12) -- (cd4_14) (cd4_13) -- (cd4_15);
\draw (cd4_8) -- (cd4_12) (cd4_9) -- (cd4_13) (cd4_10)--(cd4_14) (cd4_11)--(cd4_15);
\draw (cd4_0) -- (cd4_8) (cd4_1) -- (cd4_9) (cd4_2) -- (cd4_10) (cd4_3) -- (cd4_11) (cd4_4) -- (cd4_12) (cd4_5) -- (cd4_13) (cd4_6) -- (cd4_14) (cd4_7) -- (cd4_15);
\draw (5.5,-0.5) node{dim. 4};
\draw (2,-1) node{Hypercubes de dimension 0, 1, 2, 3 et 4};
\end{tikzpicture}
\end{center}

\end{document}

% Local Variables:
% TeX-engine: luatex
% End:
